\documentclass[12pt, twoside]{article}
\usepackage[english,russian]{babel} 
\usepackage{jmlda}
\newcommand{\hdir}{.}
%\usepackage{xcolor}

\title [Характеристики движителя подводного аппарата]
{Электромеханические характеристики движителя необитаемого подводного аппарата микрокласса}
\author[Д.\,В.~Ловчиков, О.\,Р.~Сухова]  % список авторов (не более трех) для колонтитула; не нужен, если основной список влезает в колонтитул
{Д.\,В.~Ловчиков, О.\,Р.~Сухова} % основной список авторов, выводимый в оглавление
[Д.\,В.~Ловчиков$^{1,2}$, О.\,Р.~Сухова$^2$] % список авторов, выводимый в заголовок; не нужен, если он не отличается от основного
\email{lovchikovdv@yandex.ru; i.e.sukhov1@mail.ru}
%\thanks {Работа выполнена при частичной финансовой поддержке РФФИ, проекты \No\ \No 00-00-00000 и 00-00-00001.}
\organization {$^1$МБОУ "ФМЛ №31 г. Челябинска", Челябинск; $^2$Южно-Уральский Государственный университет, Челябинск}

%\begin{document}
\abstract {Решается задача получения и анализа электромеханических характеристик движителя подводного необитаемого аппарата микрокласса. Движитель преобразует механическую энергию вращения щеточного низкотокового двигателя постоянного тока во вращение гребных винтов с помощью обеспечивающих сцепление магнитных полумуфт. Использование магнитных полумуфт вносит проблему проскальзывания винта при изменении режима управления двигателем. Построение и анализ поверхностной диаграммы ток-напряжение-обороты определяет области оптимальной совокупности электромеханических характеристик для создания автоматизированной системы обратной связи регулирования режимами управления.

\bigskip
\noindent
\textbf{Ключевые слова}: \emph {подводный движитель; электромеханические характеристики; }}

\titleEng
[JMLDA paper template] % краткое название; не нужно, если полное название влезает в~колонтитул
{Machine Learning and Data Analysis journal paper template}
\authorEng
[F.\,S.~Author] % список авторов (не более трех) для колонтитула; не нужен, если основной список влезает в колонтитул
{F.\,S.~Author, F.\,S.~Co-Author, and F.\,S.~Name} % основной список авторов, выводимый в оглавление
[F.\,S.~Author$^1$, F.\,S.~Co-Author$^2$, and F.\,S.~Name$^{1, 2}$] % список авторов, выводимый в заголовок; не нужен, если он не отличается от основного
\thanksEng
{The research was
	%partially
	supported by the Russian Foundation for Basic Research (grants 00-00-0000 and 00-00-00001).
}
\organizationEng
{$^1$Organization, address; $^2$Organization, address}
\abstractEng
{This is the template of the paper submitted to the journal ``Machine Learning and Data Analysis''.
	
	\noindent
	The title should be concise and informative. Titles are often used in information-retrieval systems. Avoid abbreviations and formulae where possible.
	Please clearly indicate the last names and initials of each author and check that all names are accurately spelled. Present the authors' affiliation
	addresses where the actual work was done.
	Provide the full postal address of each affiliation, including the country name and, if available, the
	e-mail address of each author.
	Provide only institutional affiliation, department/division affiliation are not required.
	
	\noindent
	A concise and factual abstract is required.
	The purpose of the abstract is to provide a summary~of the paper enabling the reader to decide whether or not to read the full text.
	The abstract should state briefly the purpose of the research, the principal results and major conclusions.
	An abstract is often presented separately from the article, so it must be able to stand alone.
	For this reason, References should be avoided, but if essential, then cite the author(s) and year(s).
	Also, non-standard or uncommon abbreviations should be avoided, but if essential they must be defined at their first mention in the abstract itself.
	The requirements on the size of the abstract is about 200--300 words.
	It should be provided in the next structured manner:
	
	\noindent
	\textbf{Background}:	One paragraph about the problem, existent approaches and its limitations.
	
	\noindent
	\textbf{Methods}: One paragraph about proposed method and its novelty.
	
	\noindent
	\textbf{Results}: One paragraph about major properties of the proposed method and experiment results if applicable.
	
	\noindent
	\textbf{Concluding Remarks}: One paragraph about the place of the proposed method among existent approaches.
	
	\noindent
	Immediately after the abstract, provide 5-7 keywords, avoiding general and plural terms and multiple concepts (avoid, for example, ``and'', ``of'').
	Use keywords that are specific and that reflect what is essential about the paper.
	Use keywords from the abstract, introduction and conclusion.
	These keywords will be used for indexing purposes.
	
	\noindent
	\textbf{Keywords}: \emph{keyword; keyword; more keywords, separated by ``;''}}

%данные поля заполняются редакцией журнала
\doi{10.21469/22233792}
\receivedRus{01.01.2017}
\receivedEng{January 01, 2017}

\begin{document}

\maketitle
\linenumbers
%\tableofcontents

\section{Введение}
Из-за ограниченности габаритов аккумуляторных батарей и требований по минимальным массам, на подводных аппаратах микрокласса крайне важно, чтобы движительный комплекс имел максимальный КПД и оптимальные гидродинамические обводы. При проектировании подводных аппаратов для образовательных задач становятся важными токовые характеристики электрических цепей, в том числе и потребление тока движителями. 

Подавляющее большинство действующих подводных робототехнических комплексов, как телеуправляемых так автономных, оснащено гребными движителями как основным средством маневрирования.Под движителем понимаем гребной винт, приводимый в действие электроприводом.

За последние 10 лет бесколлекторные электромоторы вытеснили другие типы двигателей, используемых на легких и сверхлегких необитаемых подводных аппаратах (НПА). При проектировании сверхлегких НПА для образовательных задач возникает потребность в компактных движителях с низким потреблением тока. Проблема отсутствия на рынке подобных решений послужила причиной создания малого движителя, работающего по принципу магнитной муфты \cite{leonov2015}. 


При разработке необитаемых подводных аппаратов важным этапом является исследования в области математического описания движителей \cite{lewis1988}. При этом движитель в целом представляет собой  нелинейную систему, характеристики которой зависят от параметров привода и гребного винта, а также метода управления \cite{kostenko2019_1, kostenko2019_2, antonenko2018, siek2019}. \textit{Производится поиск отказов движителей и в целом НАПА методами построения робастных диагностических наблюдателей для динамических систем} \cite{lukoyanov2019, filaretov2019}. Замечено, что в работах не уделяется должного внимания изучению оптимальных режимов управления движителями на магнитных полумуфтах.

Цель данной работы состоит в получении и анализе электромеханических характеристик работающего по принципу магнитной полумуфты движителя необитаемого автономного подводного аппарата микрокласса.

Объектом исследования является модель движителя с гребным винтом. Винт жестко соединен с магнитной полумуфтой, магнитосвязанной с насаженной на вал двигателя постоянного тока второй магнитной полумуфтой. Электропривод заключен в герметичный корпус. Управление осуществляется посредством подачи модулированного ШИМ-сигнала от микроконтроллера на драйвер двигателя, к которому подключен исследуемый движитель. 
Предметом исследования является семейство электромеханических характеристик движителя в швартовном режиме.

Задачи исследования:
Создание экспериментальной установки и измерительного комплекса для получения временных зависимостей тока, напряжения и числа оборотов гребного винта.
Исследование электромеханических характеристик модели движителя в швартовном режиме без проскальзывания полумуфт.
\textit{Построение математической модели используемого электропривода и гребного винта.}
Получение семейства электромеханических характеристик модели движителя в линейном режиме.

\section{Схема установки}

Установка представляет собой стенд для снятия характеристик тока, напряжения, тяги движителя и частоты оборотов внешней полумуфты.
Стенд состоит из большой емкости с водой для минимизации влияния бортов емкости на измерения, механической оснастки для фиксации движителя, комплекса снятия измерений и логирования данных. 
 Электрические характеристики снимаются с помощью монитора тока, подключенного к головному микроконтроллеру. Тяговые характеристики снимаются с помощью тензометрического моста, АЦП модуля и головного микроконтролера. Частота оборотов внешней полумуфты фиксируются с помощью датчика Холла и головного микроконтроллера. Головной микроконтроллер передает информацию по каждому из измеряемых параметров с помощью интерфейса UART порядка 1E3 раз в секунду. Полученные данные логируются для последующей математической обработки в среде Origin.
Движитель фиксируется в вертикальном положении под водой на одно из плеч равноплечего рычага. Для минимазации воздействия внешних сил рычаг подвешен на подшипнике. Второй конец рычага прикрепляется к тензометрическому мосту.
Гибкими проводами подводятся силовые кабели от регулируемого источника питания. Отдельным шлейфом идут провода датчика частоты. Токовый монитор позволяет измерить потребялемый ток и подаваемое напряжение на электродвигатель, точка подключения находится над водой в цепи питания электродвигателя.
Все вышеуказанные датчики подключены к одному головному микроконтроллеру ATMega32U4.

\section{Методика проведения измерений}

Для построения характеристических повернхностей напряжение (скважность)-ток-частота и напряжение (скважность)-ток-тяга мы придерживались следующей методики измерений. 
На ругулируемом блоке питания выставялем напряжение в диапазоне от 5 до 12 вольт с шагом 0,2 вольта. Микроконтроллер гененирует 8-битный ШИМ сигнал с шагом 1. 
Монитор тока, тензостерический мост и датчик хода фиксируют напряжение, ток, тягу и частоту оборотов внешней полумуфты и передают данные на компьютер по интерфейсу UART. 
Для каждой итерации создается отдельный csv-файл.

\bibliographystyle{gost780s}
\bibliography{LovchikovSukhova2022Underwater}

\end{document}