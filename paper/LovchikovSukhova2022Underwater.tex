\documentclass[12pt, twoside]{article}
\usepackage{jmlda}
\newcommand{\hdir}{.}
\usepackage{xcolor}
\title{Electromechanical characteristics of propulsion for micro-class underwater vehicles}
\author{Д.\,В.~Ловчиков, О.\,Р.~Сухова}
\email{lovchikovdv@yandex.ru}
\organization{Организация, Город}
\abstract{Данная статья посвящена...}
\date{}
\usepackage[latterpaper,top=2cm,bottom=2cm,left=3cm,right=1.5cm,marginparwidth=50pt]{geometry}

\begin{document}

\maketitle
\linenumbers
%\tableofcontents
\section{Abstract}
В работе решается задача получения и анализа электромеханических характеристик движителя подводного необитаемого аппарата микрокласса. Движитель преобразует механическую энергию вращения щеточного низкотокового двигателя постоянного тока (ДПТ) во вращение гребных винтов с помощью обеспечивающих сцепление магнитных полумуфт. Использование магнитных полумуфт вносит проблему проскальзывания винта при изменении режима управления двигателем. Построение и анализ поверхностной диаграммы ток-напряжение-обороты позволяет выяснить области оптимальной совокупности электромеханических характеристик для создания автоматизированной системы обратной связи регулирования режимами управления.
\section{Introduction}


\end{document}
