\documentclass[12pt, twoside]{article}
\usepackage{jmlda}
\newcommand{\hdir}{.}
\usepackage{xcolor}
\title{Electromechanical characteristics of propulsion for micro-class underwater vehicles}
\author{Д.\,В.~Ловчиков, О.\,Р.~Сухова}
\email{lovchikovdv@yandex.ru}
\organization{Организация, Город}
\abstract{Данная статья посвящена...}
\date{}
\usepackage[latterpaper,top=2cm,bottom=2cm,left=3cm,right=1.5cm,marginparwidth=50pt]{geometry}

\begin{document}

\maketitle
\linenumbers
%\tableofcontents
\section{Abstract}
Из-за ограниченности габаритов аккумуляторных батарей и требований по минимальным массам, на подводных аппаратах микрокласса крайне важно, чтобы движительный комплекс на магнитных муфтах имел максимальный КПД и оптимальные гидродинамические обводы. Существует проблема создания обладающего оптимальными электромеханическими характеристиками двигательно-движительного комплекса с системой автоматической обратной связи. Необходимо сконструировать автоматизированный измерительный стенд для получения семейства электромеханических характеристик и выбора из них оптимальных.
\section{Introduction}


\end{document}
